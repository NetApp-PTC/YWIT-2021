\chapter{Python Primer}
If you're new to Python, this section will give you a few things you should know
in order to better understand the projects in this guide. This is by no means a
complete or comprehensive look at the Python language. For that, we recommend looking
at the official Python site and reading through the \href{https://docs.python.org/3/tutorial/}{tutorial}
there.
\linebreak

\begin{tcolorbox}
    NOTE: for the projects being used here, we are using an implementation of
    Python known as \href{https://micropython.org/}{MicroPython}. This is a version
    of Python that is meant to run on microcontrollers with limited resources. It
    also has built into it libraries for dealing with hardware devices that are
    not part of the standard CPython distribution. Therefore, not all Python examples
    you find online will run on your microcontroller and not all projects for a
    microcontroller can be run on your computer. But a lot of the code can be shared
    so the lessons you learn here can apply to other Python projects.
\end{tcolorbox}

Here is a sample of a small Python script. We will disect and explain what each
section does below:

\begin{lstlisting}[language=Python,caption=An example Python script]
def show(message, repeat=1):
    """This function prints the given message to the console as many
    times as specified in the repeat parameter.
    """

    for iteration in range(0, repeat):
        print(message)

name = input("What is your name: ")
show(name)
show(name, repeat=2)
\end{lstlisting}

On line 1, we are defining a function named show.